% Options for packages loaded elsewhere
\PassOptionsToPackage{unicode}{hyperref}
\PassOptionsToPackage{hyphens}{url}
%
\documentclass[
]{article}
\usepackage{amsmath,amssymb}
\usepackage{iftex}
\ifPDFTeX
  \usepackage[T1]{fontenc}
  \usepackage[utf8]{inputenc}
  \usepackage{textcomp} % provide euro and other symbols
\else % if luatex or xetex
  \usepackage{unicode-math} % this also loads fontspec
  \defaultfontfeatures{Scale=MatchLowercase}
  \defaultfontfeatures[\rmfamily]{Ligatures=TeX,Scale=1}
\fi
\usepackage{lmodern}
\ifPDFTeX\else
  % xetex/luatex font selection
\fi
% Use upquote if available, for straight quotes in verbatim environments
\IfFileExists{upquote.sty}{\usepackage{upquote}}{}
\IfFileExists{microtype.sty}{% use microtype if available
  \usepackage[]{microtype}
  \UseMicrotypeSet[protrusion]{basicmath} % disable protrusion for tt fonts
}{}
\makeatletter
\@ifundefined{KOMAClassName}{% if non-KOMA class
  \IfFileExists{parskip.sty}{%
    \usepackage{parskip}
  }{% else
    \setlength{\parindent}{0pt}
    \setlength{\parskip}{6pt plus 2pt minus 1pt}}
}{% if KOMA class
  \KOMAoptions{parskip=half}}
\makeatother
\usepackage{xcolor}
\usepackage[margin=1in]{geometry}
\usepackage{longtable,booktabs,array}
\usepackage{calc} % for calculating minipage widths
% Correct order of tables after \paragraph or \subparagraph
\usepackage{etoolbox}
\makeatletter
\patchcmd\longtable{\par}{\if@noskipsec\mbox{}\fi\par}{}{}
\makeatother
% Allow footnotes in longtable head/foot
\IfFileExists{footnotehyper.sty}{\usepackage{footnotehyper}}{\usepackage{footnote}}
\makesavenoteenv{longtable}
\usepackage{graphicx}
\makeatletter
\def\maxwidth{\ifdim\Gin@nat@width>\linewidth\linewidth\else\Gin@nat@width\fi}
\def\maxheight{\ifdim\Gin@nat@height>\textheight\textheight\else\Gin@nat@height\fi}
\makeatother
% Scale images if necessary, so that they will not overflow the page
% margins by default, and it is still possible to overwrite the defaults
% using explicit options in \includegraphics[width, height, ...]{}
\setkeys{Gin}{width=\maxwidth,height=\maxheight,keepaspectratio}
% Set default figure placement to htbp
\makeatletter
\def\fps@figure{htbp}
\makeatother
\ifLuaTeX
  \usepackage{luacolor}
  \usepackage[soul]{lua-ul}
\else
  \usepackage{soul}
\fi
\setlength{\emergencystretch}{3em} % prevent overfull lines
\providecommand{\tightlist}{%
  \setlength{\itemsep}{0pt}\setlength{\parskip}{0pt}}
\setcounter{secnumdepth}{5}
\usepackage[normalem]{ulem}
\ifLuaTeX
  \usepackage{selnolig}  % disable illegal ligatures
\fi
\usepackage{bookmark}
\IfFileExists{xurl.sty}{\usepackage{xurl}}{} % add URL line breaks if available
\urlstyle{same}
\hypersetup{
  pdftitle={Curso introductorio a QGIS aplicado a la sismología y geología},
  pdfauthor={José Ramón Martínez-Batlle},
  hidelinks,
  pdfcreator={LaTeX via pandoc}}

\title{Curso introductorio a QGIS aplicado a la sismología y geología}
\author{José Ramón Martínez-Batlle}
\date{2026-02-21}

\begin{document}
\maketitle

{
\setcounter{tocdepth}{2}
\tableofcontents
}
\section{Código QR de este repo}\label{cuxf3digo-qr-de-este-repo}

\includegraphics{qr.jpg}

\section{Introducción}\label{introducciuxf3n}

Este curso tiene como objetivo capacitar a un grupo reducido de participantes (geólogos y sismólogos) en el uso de \textbf{QGIS} para la elaboración de mapas de sismos y temáticas geológicas, tanto en 2D como en 3D.

La duración total estimada es de \textbf{3 días (jornadas de 4 horas, total 12 horas)}.

Se combinarán exposiciones breves con ejercicios prácticos reproducibles, de manera que los participantes adquieran destrezas aplicables a su trabajo cotidiano.

\begin{center}\rule{0.5\linewidth}{0.5pt}\end{center}

\section{Objetivos}\label{objetivos}

\begin{itemize}
\tightlist
\item
  Aprender a cargar, visualizar y organizar datos georreferenciados en QGIS (vectoriales, ráster, tablas con coordenadas). - Generar mapas de sismos clasificados por magnitud, profundidad y otros parámetros.
\item
  Construir curvas de nivel y superficies de terreno a partir de modelos digitales de elevación (DEM).
\item
  Explorar opciones de visualización 3D y elaboración de perfiles topográficos.
\item
  Usar simbología especializada (SVG del USGS, simbología geológica y sismológica).
\item
  Introducir el manejo de \textbf{GeoPackages} y prácticas para compartir información espacial.
\item
  Explorar brevemente el uso de scripting en Python (PyQGIS) y herramientas complementarias en R/Python para análisis espacial avanzado.
\end{itemize}

\begin{center}\rule{0.5\linewidth}{0.5pt}\end{center}

\section{Fuente de datos}\label{fuente-de-datos}

\href{https://drive.google.com/drive/folders/19EwWTFqjO99mUOBab8qtcBImLF_MQ1Dq?usp=drive_link}{Visitar esta carpeta Drive}

\begin{center}\rule{0.5\linewidth}{0.5pt}\end{center}

\section{Programa del curso}\label{programa-del-curso}

\subsection{Día 1. Fundamentos de QGIS y manejo y visualización de datos geoespaciales}\label{duxeda-1.-fundamentos-de-qgis-y-manejo-y-visualizaciuxf3n-de-datos-geoespaciales}

\href{https://geofis.github.io/curso-qgis-tecnologias-geoespaciales-ipl-2026/dia1.html}{Presentación de diapositivas}

\subsubsection{Primera parte. Fundamentos de QGIS y manejo de datos geoespaciales}\label{primera-parte.-fundamentos-de-qgis-y-manejo-de-datos-geoespaciales}

\begin{itemize}
\tightlist
\item
  \textbf{Instalación y entorno QGIS}

  \begin{itemize}
  \tightlist
  \item
    Revisión de la interfaz gráfica.
  \item
    Panel de capas, navegador y estilos.
  \end{itemize}
\item
  \textbf{Fuentes de datos}

  \begin{itemize}
  \tightlist
  \item
    Vectoriales (Shapefile, GeoPackage).
  \item
    Ráster (GeoTIFF, WMS/WMTS).
  \item
    Tablas con coordenadas (CSV/TSV).
  \end{itemize}
\item
  \textbf{Ejercicio práctico:}

  \begin{itemize}
  \tightlist
  \item[$\boxtimes$]
    \st{Cargar archivos vectoriales shapefiles/geopackages de las tres primitivas comunes: puntos, líneas y polígonos.}
  \item[$\boxtimes$]
    \st{Cargar archivos rásters.}
  \item
    Cargar un conjunto de eventos sísmicos desde CSV (coordenadas lat/long).
  \item
    Cargar una fuente WMS.
  \item
    Visualización básica en mapa, composicion.
  \item
    Hablemos de OSGeoLive (\url{https://live.osgeo.org}).
  \item
    Sesión de retroalimentación, problemas aportados por participantes.
  \end{itemize}
\end{itemize}

\begin{center}\rule{0.5\linewidth}{0.5pt}\end{center}

\subsection{Dia 2. Atributos, simbología, georreferenciación. Intro a modelos de terreno, curvas de nivel y visualización 3D}\label{dia-2.-atributos-simbologuxeda-georreferenciaciuxf3n.-intro-a-modelos-de-terreno-curvas-de-nivel-y-visualizaciuxf3n-3d}

\subsubsection{Primera parte. Atributos, simbología y georreferenciación}\label{primera-parte.-atributos-simbologuxeda-y-georreferenciaciuxf3n}

\begin{itemize}
\tightlist
\item
  \textbf{Ejercicio práctico:}

  \begin{itemize}
  \item
    \textbf{Tabla de atributos}

    \begin{itemize}
    \tightlist
    \item
      Creación de tabla de atributos.
    \item
      Edición y gestión de atributos.
    \end{itemize}
  \item
    \textbf{Calculadora de campos}

    \begin{itemize}
    \tightlist
    \item
      Creación de campos calculados (ejemplo: días desde el sismo más antiguo).
    \end{itemize}
  \item
    \textbf{Selección por atributos.}

    \begin{itemize}
    \tightlist
    \item
      Vía tabla de atributos.

      \begin{itemize}
      \tightlist
      \item
        (``date''\textless=`2020/2/20')
      \item
        (``date''\textgreater{}`2020/2/20')
      \end{itemize}
    \item
      Vía simbología.
    \end{itemize}
  \item
    \textbf{Selección por ubicación.}
  \item
    \textbf{Calculadora de campos}. Clasificación por campos calculados

\begin{verbatim}
if(
  "date" IS NULL,
  NULL,
  day(
    "date" - minimum("date", filter:= "date" IS NOT NULL)
  )
)
\end{verbatim}
  \item
    \textbf{Simbología}

    \begin{itemize}
    \tightlist
    \item
      Clasificación por atributos (magnitud, profundidad).
    \item
      Uso de simbología SVG del USGS para mapas geológicos.

      \begin{itemize}
      \tightlist
      \item
        Vía plugin: \url{https://qgis-in-mineral-exploration.readthedocs.io/en/latest/source/how_to/USGS.html}.
      \item
        Fuente de símbolos: \url{https://github.com/rodreras/geologic_icons}.
      \item
        Ejemplo de uso: \url{https://geofis.xyz/lm/index.php/view/map/?repository=geo250krd&project=geologico_gpkg}
      \end{itemize}
    \end{itemize}
  \item
    \textbf{Etiquetado y composición de mapas}

    \begin{itemize}
    \tightlist
    \item
      Configuración de leyendas, escalas gráficas y títulos.
    \end{itemize}
  \item
    \textbf{Georreferenciación}

    \begin{itemize}
    \tightlist
    \item
      Uso del georreferenciador de QGIS (para ráster y mapas antiguos).
    \item
      Georreferenciar un mapa geológico escaneado.
    \item
      Cargar un GeoPDF del SGN, convertirlo a ráster para mayor eficiencia de despliegue.
    \item
      Georreferenciar un mapa vectorial.
    \item
      Superponerlo a la capa de sismos.
    \item
      Sesión de retroalimentación, problemas aportados por participantes.
    \end{itemize}
  \end{itemize}
\end{itemize}

\subsubsection{Segunda parte. Intro modelos de terreno y curvas de nivel}\label{segunda-parte.-intro-modelos-de-terreno-y-curvas-de-nivel}

\begin{itemize}
\tightlist
\item
  \textbf{Fuentes de modelos digitales de elevación (DEM)}

  \begin{itemize}
  \tightlist
  \item
    Descarga (Copernicus, SRTM).
  \item
    Carga desde archivos o servicios web.\\
  \end{itemize}
\item
  \textbf{Generación de isolínes/curvas de nivel}

  \begin{itemize}
  \tightlist
  \item
    A partir de DEM (herramienta de procesamiento QGIS/GDAL).\\
  \item
    ¿Qué pasaría si necesito curvas de nivel (o isolíneas en general) a partir de puntos?
  \end{itemize}
\item
  \textbf{Perfiles topográficos}

  \begin{itemize}
  \tightlist
  \item
    Uso de \emph{Profile tool} o herramienta nativa de QGIS.\\
  \end{itemize}
\item
  \textbf{Ejercicio práctico:}

  \begin{itemize}
  \tightlist
  \item
    Crear curvas de nivel cada 20 m en un área de interés.
  \item
    Cargar fuentes WMS que contengan las curvas de nivel.
  \item
    Comparar perfiles topográficos de zonas epicentrales.
  \item
    Sesión de retroalimentación, problemas aportados por participantes.
  \end{itemize}
\end{itemize}

\begin{center}\rule{0.5\linewidth}{0.5pt}\end{center}

\subsubsection{Tercera parte. Visualización 3D y análisis espacial}\label{tercera-parte.-visualizaciuxf3n-3d-y-anuxe1lisis-espacial}

\begin{itemize}
\tightlist
\item
  \textbf{Ventana 3D de QGIS}

  \begin{itemize}
  \tightlist
  \item
    Configuración de terreno a partir de DEM.
  \item
    Extrusión de capas vectoriales.
  \item
    Exploración de estabilidad y limitaciones.
  \end{itemize}
\item
  \textbf{Análisis espacial básico}

  \begin{itemize}
  \tightlist
  \item
    Buffer de epicentros.
  \item
    Densidad de puntos (heatmaps).
  \end{itemize}
\item
  \textbf{Ejercicio práctico:}

  \begin{itemize}
  \tightlist
  \item
    Visualizar un mapa 3D con epicentros clasificados por profundidad.
  \item
    Generar un mapa de calor de sismos en 2D.
  \end{itemize}
\end{itemize}

\begin{center}\rule{0.5\linewidth}{0.5pt}\end{center}

\subsection{Día 3. Extensiones, scripting y compartición de datos}\label{duxeda-3.-extensiones-scripting-y-comparticiuxf3n-de-datos}

\subsubsection{Primera parte. Extensiones y análisis avanzado}\label{primera-parte.-extensiones-y-anuxe1lisis-avanzado}

\begin{itemize}
\tightlist
\item
  \textbf{Geoestadística e interpolación}

  \begin{itemize}
  \tightlist
  \item
    Breve introducción a superficies continuas (GRASS GIS, SAGA, ambos desde QGIS).
  \end{itemize}
\item
  \textbf{Scripting en QGIS (PyQGIS)}

  \begin{itemize}
  \tightlist
  \item
    Uso de la consola de Python en QGIS.
  \item
    Ejemplo: filtrar sismos por magnitud y exportar resultados.
  \end{itemize}
\item
  \textbf{Herramientas externas}

  \begin{itemize}
  \tightlist
  \item
    R: análisis espacial, geoestadística, patrones de sismos.
  \item
    Python: una mirada (``muy por encima) a geopandas, obspy, matplotlib para análisis sísmico.
  \end{itemize}
\item
  \textbf{Compartición de datos}

  \begin{itemize}
  \tightlist
  \item
    Guardado en GeoPackage.
  \item
    Compartir en la nube.
  \item
    HTML + JS: Biblioteca leaflet.
  \item
    Breve mención a servidores y alternativas de publicación (QGIS Server, Lizmap).\\
  \end{itemize}
\item
  \textbf{Ejercicio práctico:}

  \begin{itemize}
  \tightlist
  \item
    Exportar un proyecto a GeoPackage y compartirlo.\\
  \item
    Generar un script simple en PyQGIS para automatizar un mapa temático.
  \end{itemize}
\end{itemize}

\begin{center}\rule{0.5\linewidth}{0.5pt}\end{center}

\section{Metodología}\label{metodologuxeda}

\begin{itemize}
\tightlist
\item
  Exposiciones breves (20--30 min) con ejemplos prácticos.\\
\item
  Ejercicios guiados paso a paso.\\
\item
  Espacio para preguntas y exploración autónoma de datos.\\
\item
  Recomendación de recursos abiertos (manuales, plugins, datasets).
\end{itemize}

\begin{center}\rule{0.5\linewidth}{0.5pt}\end{center}

\section{Evaluación y cierre}\label{evaluaciuxf3n-y-cierre}

\begin{itemize}
\tightlist
\item
  No habrá evaluación formal.\\
\item
  Se solicitará a cada participante elaborar un \textbf{mapa final} de sismos con curvas de nivel y simbología geológica, y (opcionalmente) una visualización 3D.
\end{itemize}

\begin{center}\rule{0.5\linewidth}{0.5pt}\end{center}

\section{Bibliografía sugerida}\label{bibliografuxeda-sugerida}

\begin{itemize}
\tightlist
\item
  Wu, Qiusheng (2025). Introducción a la Programación GIS Una Guía Práctica de Python para Herramientas Geoespaciales de Código Abierto. \url{https://leanpub.com/gispro-es}
\item
  QGIS Documentation: \url{https://docs.qgis.org}
\item
  Olaya, V. (2020). Sistemas de Información Geográfica. Libro SIG
\item
  Hengl, T. (2009). \emph{A Practical Guide to Geostatistical Mapping}.\\
\item
  Dorman, M., Graser, A., Nowosad, J., \& Lovelace, R. (2025). Geocomputation with Python. CRC Press. \url{https://py.geocompx.org/}
\item
  Lovelace, R., Nowosad, J., \& Muenchow, J. (2019). Geocomputation with R. CRC Press. \url{https://r.geocompx.org/}
\item
  USGS Symbol Library:

  \begin{itemize}
  \tightlist
  \item
    \url{https://github.com/afrigeri/geologic-symbols-qgis}
  \item
    \url{https://github.com/BC-Consulting/FGDC-4-QGIS}
  \item
    \url{https://github.com/rodreras/geologic_icons}
  \item
    \url{https://davenquinn.com/projects/geologic-patterns/}
  \item
    \url{https://sourceforge.net/projects/qgisgeologysymbology/}
  \end{itemize}
\item
  Obspy: A Python Toolbox for Seismology (\url{https://docs.obspy.org})\\
\item
  Geopandas: Python tools for geographic data (\url{https://geopandas.org})
\item
  Leaflet: An open-source JavaScript library for mobile-friendly interactive maps (\url{https://leafletjs.com})
\end{itemize}

\end{document}
